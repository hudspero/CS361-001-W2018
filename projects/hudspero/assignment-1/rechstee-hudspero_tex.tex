\documentclass[12pt, letterpaper]{article}
\usepackage{hyperref}
%opening
\title{Vision Statement}
\author{rechstee-hudspero}

\begin{document}

\maketitle Project name: Wat2Search, Team:Team Chronic
\begin{enumerate}
	\item \textbf{Your team name, team onid, and project name (try to make up a short,catchy name for your project).}
	
	We are Team Chronic, comprised of users rechstee and hudspero, and our project is Wat2Search IT (working title).
	
	\item \textbf{What’s a problem you have noticed that someone might be encountering in the world?}
	
	A problem that we have noticed to be quite present is that novice to intermediate PC users don’t know how to accurately describe what kind of problem they’re having with their computers to the technicians that’re trying to help them.
	
	\item \textbf{What evidence do you have about this problem—did you read it in the news,see it on the street, or read it in the research?}
	
	We both have extensive experience working for the Service Desk in OSU IT, Community Network, and have created multiple customer tickets spanning years on this issue. 
	
	\item \textbf{What are   the   details   of   the problem (e.g., how often does it happen? To whom? Why?)?}
	
	Anyone who owns a computer, and doesn’t have much knowledge about how to use it in any capacity, is susceptible to this issue. Students and older faculty, in particular, are far more likely to face this issue, and it happens more often than you’d think.
	
	\item \textbf{Cite sources for your evidence.}
	
	\begin{itemize}
	\item \url{https://www.reddit.com/r/talesfromtechsupport}
	\item \url{https://www.reddit.com/r/talesfromtechsupport/comments/63rh18/thats_not_how_that_works/}
	\item \url{https://www.reddit.com/r/talesfromtechsupport/comments/49ctxn/all_of_my_computers_are_wireless_ive_said_it_5/}
	\item \url{https://www.reddit.com/r/talesfromtechsupport/comments/5okg2b/everyone_is_wrong_except_me/}
	\item \url{https://www.reddit.com/r/talesfromtechsupport/comments/56mtz2/all_of_my_email_is_gone_someone_deleted_all_of_it/}
	\end{itemize}
	
	
	\item \textbf{Tell a brief story of what happens when someone encounters this problem in real life/real time.}
	
	When a customer calls the Service Desk, or an IT technician, the main goals are to diagnose the issue(s), and find the appropriate solution(s). If the customer doesn’t tell us the correct information at first, or insists that their issue lies in something different than what we’ve diagnosed it to be, then the technicians have to start from the beginning. This means that all of the effort spent fixing a non-existent problem was wasted, and that’s time both the user and the technician will never get back, which slows down the service as a whole for the other customers waiting their turn, and the other technicians involved.
	
	\item \textbf{How might a software system or application/approach affect/solve the problem?}
	
	A simple, interactive, graphical flow-chart/tree is a simple and effective way of simply diagnosing an issue, without needing an Internet connection or phone call. It saves time for the technicians so they can fix the issue, and it helps educate the customer so they know what’s going on with their system. 
	
	\item \textbf{State the key difference between your approach and previous approaches.}

	The only alternative to our solution is to not have a chart at all, and to rely on the technicians to explain everything over the phone or text. 

	\item \textbf{State limitations, if any, of your application/approach.}
	
	The biggest limitation of this solution is the end users themselves, which we as technicians and developers have absolutely no solution for. Some users are stubborn, impatient, and selfish enough to think that the world revolves around them. These users don’t care what their issue is, so long they can their ego remains untouched. Other users are “technologically-allergic”, and absolutely refuse to touch anything with circuitry. They’re aware that there’s a problem, but don’t want to learn about it, and expect the technicians to push a magic button to fix it.
	
	\item \textbf{The resources you will need to complete the project. This includes software, or hardware, documentations or descriptions of database you propose to access, etc.}
	
	Most resources we’ll need to use this software can be found online or come standard with Microsoft Office, or on the OSU EECS server. Graphics, lettering, and texturing can be done through PowerPoint/Paint, and code can be written and tested on the OSU EECS ENGR server.
	
	\item \textbf{What is the single most serious challenge you see in developing the product on schedule? How will you minimize or mitigate the risk?. Be specific and concrete. Don't give generic risks that would be equally applicable to any project. You might give the system architecture, describing at a very high level the components that will interact in your system along with existing components you might reuse.}	
	
	The biggest challenge I can see when designing this program is being able to allocate time correctly for the project. 
	
\end{enumerate}


\end{document}
