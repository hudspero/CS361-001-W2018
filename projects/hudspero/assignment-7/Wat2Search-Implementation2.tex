\documentclass[12pt, letterpaper]{article}
\usepackage{hyperref}
%opening
\title{Wat2Search-Implementation 2}
\author{Evan Rechsteiner rechstee, Robert Hudspeth hudspero, Donghao Lin lindo, \\Luke Miletta milettal, Lauren Sunamoto sunamotl 
}
\begin{document}
	\maketitle
	\section{Product Release:}
	people.oregonstate.edu/\textasciitilde hudspero
	\section{User Stories:}
	   \subsection{Accessing information on Transcripts}
	   \begin{enumerate}
	   	\item \emph{Which pair(s) of teammates worked on that user story's tasks?}
	   	\\Robert and Luke
	   	\item \emph{What problems, if any, did you encounter?}
	   	\\No serious problems were encountered, only small concerns over the comprehension of instructions steps.
	   	\item \emph{How long did each task require?}
	   	\\About 30 minutes.  Although the first task of creating a basic UI was completed last week.
	   	\item \emph{What is the current status (implemented? tested?)}
	   	\\This user story has already been tested and implemented and therefore is fully completed.
	   	\item \emph{What is left to be completed?}
	   	\\ As noted above there is nothing left to be completed concerning this specific user story.
	   	\item \emph{For each spike and UML sequence diagram that you developed this week, answer the following}
	   	\begin{itemize}
	   		\item \emph{Was the spike or diagram useful? Why or why not?}
	   		\\ As noted above there is nothing left to be completed concerning this specific user story.
	   		\item \emph{Were there any diagrams that you wish that you had? Why or why not?}
	   		\\Any diagrams were not required as the spike was what we relied on mostly for implementation.
	   	\end{itemize}
		\end{enumerate}
	   
	   \subsection{Information on program licenses available to students}
	   \begin{enumerate}
	   	\item \emph{Which pair(s) of teammates worked on that user story's tasks?}
	   	\\Robert and Luke
	   	\item \emph{What problems, if any, did you encounter?}
	   	\\ For this user case, the task was pretty straightforward and therefore we did not encounter any serious issues during coding.  The only issue we faced was recognizing any ambiguity in the wording of instructions that may confuse the user.
	   	\item \emph{How long did each task require?}
	   	\\The first task of thorough researching the applications offered by Oregon State university took a couple hours.  The second task took about one and a half hours.
	   	
	   	\item \emph{What is the current status (implemented? tested?)}
	   	\\This user story has already been tested and implemented and therefore is fully completed.
	   	\item \emph{What is left to be completed?}
	   	\\  As noted above there is nothing left to be completed concerning this specific user story.
	   	\item \emph{For each spike and UML sequence diagram that you developed this week, answer the following}
	   	\begin{itemize}
	   		\item \emph{Was the spike or diagram useful? Why or why not?}
	   		\\The spike was helpful in which dealing with an individual page layout and design is better than designing the whole website one way and having to change each individual page later on to fit its informational needs. 
	   		\item \emph{Were there any diagrams that you wish that you had? Why or why not?}
	   		\\Any diagrams were not required as the spike was what we relied on mostly for implementation.
	   	\end{itemize}
	   \end{enumerate}
   
   \subsection{Direct users to step by step guides in documentation where possible}
   \begin{enumerate}
   	\item \emph{Which pair(s) of teammates worked on that user story's tasks?}
   	\\Rob and Luke
   	\item \emph{What problems, if any, did you encounter?}
   	\\For this user case, the task was pretty straightforward and therefore we did not encounter any serious issues during coding.  The only issue we faced was recognizing any ambiguity in the wording of the step instructions that may confuse the user.
   	\item \emph{How long did each task require?}
   	\\Under 1 hour
   	\item \emph{What is the current status (implemented? tested?)}
   	\\This user story has already been tested and implemented and therefore is fully completed.
   	\item \emph{What is left to be completed?}
   	\\As noted above there is nothing left to be completed concerning this specific user story.
   	\item \emph{For each spike and UML sequence diagram that you developed this week, answer the following}
   	\begin{itemize}
   		\item \emph{Was the spike or diagram useful? Why or why not?}
   		\\The spike was somewhat helpful in ensuring the users have a clean user experience and do not encounter difficulties while navigating the application.
   		\item \emph{Were there any diagrams that you wish that you had? Why or why not?}
   		\\Any diagrams were not required as the spike was what we relied on mostly for implementation.
   	\end{itemize}
   \end{enumerate}

\subsection{Subcategory of OSU apps under Accounts}
\begin{enumerate}
	\item \emph{Which pair(s) of teammates worked on that user story's tasks?}
	\\Rob and Luke
	\item \emph{What problems, if any, did you encounter?}
	\\No serious problems were encountered.
	\item \emph{How long did each task require?}
	\\About half an hour
	\item \emph{What is the current status (implemented? tested?)}
	\\Implemented in a test environment but programmed in.
	\item \emph{What is left to be completed?}
	\\This user story has already been tested and implemented and therefore is fully completed.
	\item \emph{For each spike and UML sequence diagram that you developed this week, answer the following}
	\begin{itemize}
		\item \emph{Was the spike or diagram useful? Why or why not?}
		\\The spike was not required as this specific user story was very straightforward.
		\item \emph{Were there any diagrams that you wish that you had? Why or why not?}
		\\No diagrams were required because this user story was fairly simple and did not need thorough testing.
	\end{itemize}
\end{enumerate}

\subsection{Styling done in CSS}
\begin{enumerate}
	\item \emph{Which pair(s) of teammates worked on that user story's tasks?}
	\\Robert and Luke.
	\item \emph{What problems, if any, did you encounter?}
	\\For this user case, the task was pretty straightforward and therefore we did not encounter any serious issues.  The only issues we faced were easily and quickly resolved such as a coding typo.
	\item \emph{How long did each task require?}
	\\About 1 ½ hours.
	\item \emph{What is the current status (implemented? tested?)}
	\\This user story has already been tested and implemented and therefore is fully completed.
	\item \emph{What is left to be completed?}
	\\There is nothing left to be addressed for completion concerning this specific user story.
	\item \emph{For each spike and UML sequence diagram that you developed this week, answer the following}
	\begin{itemize}
		\item \emph{Was the spike or diagram useful? Why or why not?}
		\\This helped somewhat to ensure that the CSS is good for the program and is visually appealing. A non-visually appealing app will have a negative impact on the user and they will be less likely to use the app, which takes away from its usefulness.
		\item \emph{Were there any diagrams that you wish that you had? Why or why not?}
		\\There were not any diagrams that we wished we had because we already had some pretty useful paper prototypes.
	\end{itemize}
\end{enumerate}

	
	
	
	\section{Design Changes and Rationale}
For this assignment we decided to not meet in person and instead split up tasks for each individual to accomplish in their own time.  We were still able to easily communicate through the Discord app and efficiently divide the assignment among each group member.  This decision to not meet up and work separately was made because we all had a solid understanding of our expectations for the project and program and therefore participation in collaboration was not necessary.  This is in large part to good planning in past meetings and constant communication over various aspects of the project ranging from design elements to user comprehension and usability.  Furthermore, the ability to work on the project individually was convenient in which any conflicting schedules was resolved.  In regards to programming, we followed our schedule and were able to complete all of the user stories we planned to finish by the week.  
Throughout the past week of programming there were not any significantly important questions we needed to ask our customers.  But, all and any questions asked received quick responses that helped clarify any expectations the customer had concerning the programming implementation and the project’s final product.  A common type of question asked often related to the usability of the product in which we wanted to ensure things like instruction wording or means of navigability were designed in mind of our target users.  An example of question would be, “What kind of image should we use for the category, “Hardware” when determining the icons to be implemented for users that benefit from visuals.  In response, the customer responded by providing their opinion that a computer icon would be best understood by most people.

\section{Refactoring}
Overall a lot of the code stayed the same but for the Javascript portion we did wind up realizing that Javascript functions have to start with underscores. Really less of reorganizing and more just a small rule we forgot about. Something that changed was the addition of having Javascript in the program, the functions were still intended from the start but it does mean in the same program file you add the script up above with a script search function built into html.  Our project is pretty straight forward and any reiterating on how to navigate from category to category was already established even before working on the program. The logic that yields the results we desired overall stayed the same for us. The HTML pretty much stayed the same since it is just a place to hold tables with the different categories. We’d probably need another week to go back and notice any changes in efficiency in our non functional requirements for the Javascript. I would still stand by the HTML being fine though in how it organizes content.

	\section{Tests}
This week, the program is much more implemented than last week. There is a “start over” at the bottom of the pages and more features are completed. 
There are 8 icons on the homepage. 
When clicking “Registration” , there are 3 options on the screen, “How do I sign up for classes?”, “I'm getting an error message registering for classes”, “How do I get rid of holds?”. But there is no reaction when click the option above. And then clicking “start over” button to return to homepage.
For testing account troubleshooting tree, we click “Account” button on the home page, then there are 5 options, “new students”, “current students”, “former students”, “students employees” and “other employees”.  This specific user story is not completely implemented, so only “”current students” can be clicked and redirect to another page. And then, there are two options shown, “How do I change my ONID password?”, and “How do I set up/change my alternate contact information?”.  When clicking on the first option, “How do I change my ONID password?”, it turns to another page that contains a solution.
When clicking “Hardware”, there are 2 options on the screen, “CN-supported Devices” and “Personal Devices”, and a “start over” button at the bottom. But the functions are not implemented for those 2 options, so nothing happened when clicking them.
When clicking “Software”, there are 5 options on the screen, “Microsoft Office”, “Microsoft Outlook and Exchange”, “Web browsers (Chrome, Firefox, Edge, Safari)”, “Citrix and apps.oregonstate.edu”, “Other software”, and a “start over” button at the bottom. But the functions are not implemented for those 5 options, so nothing happened when clicking them. 
When clicking “Mobile Devices”, there are 3 options on the screen, “I need to connect to Wi-Fi”, “I need to set up my Gmail/Outlook”, “How do I print from my phone?”, and a “start over” button at the bottom. But the functions are not implemented for those 3 options, so nothing happened when clicking them. 
When clicking “Printers and Monitors”, there are 2 options on the screen, “Printers” and “Monitors”. But only “Printers” button is implemented, when clicking it, it goes to a page with 2 options, “Departmental/Office Printers”, and “Personal Printers”. There is not reaction when clicking “Departmental/Office Printers”, and it shows a solution when clicking “Personal Printers”.
When clicking “Networking and Wi-Fi” button, then there are four options shown on the screen. These options are either consecutive categories or questions that help the use navigate and discern the specific problem they need help with.  The four options are respectively: “Help with wireless networks”, “Help with wired network”, “Where is the internet down” and “Connect to network drive”. When clicking the first option which is “help with wireless networks”, the website redirects to another step, there are also four options, “My wireless connection is unstable”, “My computer/phone lost connection to the internet”, “I want to setup a wireless network”, “What is VPN? How do I set it up?”. Only the option “I want to setup a wireless network” returns a solution.


\section{Meeting Report}
Given that the nature of this assignment has us working off of documents we already had made we decided to contribute remotely. We continued communication through Discord, which worked well enough to relay information. Given we had this assignment, the presentation, as well as finishing a viable build of the program this method seemed to work best given how scattered our efforts had to be. Robert and Luke put a lot more time into the program and we all tested it out. We agreed that it was imperative to move on to the written and presentation so that we can practice and remain solid in our understanding of the program.For the presentation we referred to the previous presentation to compare it to the requirements of assignment 8. For the written portion we split the work similarly to before. Robert also did a lot of work on the presentation. Evan did some coordinating, helped with various written portions, the LaTex and meeting report.Luke and Lauren worked on the sites graphics. Donghao and Lauren also contributed to Implementation 2. 
This is the final week, we are working towards showing off a polished product this upcoming week. For the presentation we will all be there and will go over what progress we made since the SDSUI assignment to turn it into a functional program. In this assignment we agreed upon doing the user stories that were left and for the program we wanted to reflect that by having the user stories also fully functional. We’ll see what the next week brings, but we pretty much have a program that is presentable for a pitch. Our focus for the final week, will be acing our tests. 






\end{document}