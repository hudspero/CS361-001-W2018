\documentclass[12pt, letterpaper]{article}
\usepackage{hyperref}
%opening
\title{Wat2Search-Implementation 1}
\author{Evan Rechsteiner rechstee, Robert Hudspeth hudspero, Donghao Lin lindo, \\Luke Miletta milettal, Lauren Sunamoto sunamotl 
}
\begin{document}
	\maketitle
	\section{User Stories:}
	   \subsection{Account troubleshooting tree:}
	   \begin{enumerate}
	   	\item \emph{Which pair(s) of teammates worked on that user story's tasks?}
	   	\\Luke and Robert. They met up for programming on Sunday. We all discussed as a team as well.
	   	\item \emph{What problems, if any, did you encounter?}
	   	\\Just determining how to present this branch by referring to what we wrote in previous assignments. 
	   	\item \emph{How long did each task require?}
	   	\\½ hour
	   	\item \emph{What is the current status (implemented? tested?)}
	   	\\Since this is a full branch of the system we are waiting until next week before fully implementing it. Currently we just have the full implementation of one of our test cases.
	   	\item \emph{What is left to be completed?}
	   	\\We need to finalize implementation of the branches that come from this main one. 
	   	\item \emph{For each spike and UML sequence diagram that you developed this week, answer the following}
	   	\begin{itemize}
	   		\item \emph{Was the spike or diagram useful? Why or why not?}
	   		\\This is a more straight forward feature but will still take a lot of work when 
	   		\item \emph{Were there any diagrams that you wish that you had? Why or why not?}
	   		\\I think a whiteboard and referring to our previous UML diagrams will be enough for us to finalize the flow.
	   	\end{itemize}
		\end{enumerate}
	   
	   \subsection{Network troubleshooting tree:}
	   \begin{enumerate}
	   	\item \emph{Which pair(s) of teammates worked on that user story's tasks?}
	   	\\Same as the previous one. We all had some involvement testing but Rob and Luke are the primary programming team.
	   	\item \emph{What problems, if any, did you encounter?}
	   	\\ Same as before, adding the feature took looking at the previous assignments and we just had to verify that we were comfortable with.
	   	\item \emph{How long did each task require?}
	   	\\½ hour
	   	\item \emph{What is the current status (implemented? tested?)}
	   	\\User story is pretty much implemented if user hits everything needed for this problem.
	   	\item \emph{What is left to be completed?}
	   	\\Finalizing the final solutions and questions that stem from this branch.
	   	\item \emph{For each spike and UML sequence diagram that you developed this week, answer the following}
	   	\begin{itemize}
	   		\item \emph{Was the spike or diagram useful? Why or why not?}
	   		\\Same as User Story 1.This is a more straight forward feature but will still take a lot of work when actually added so we didn’t refer to a spike. It’s more based on principle and belief of what yields the most simplistic branch. 
	   		\item \emph{Were there any diagrams that you wish that you had? Why or why not?}
	   		\\Same as User Story 1 but I wish there were more tutorials on how to mimic other widgets with wireless network IT support. It’s awkward trying to convey the troubleshooting through text without relying on tech jargon.
	   	\end{itemize}
	   \end{enumerate}
   
   \subsection{Create as a web widget:}
   \begin{enumerate}
   	\item \emph{Which pair(s) of teammates worked on that user story's tasks?}
   	\\Rob and Luke
   	\item \emph{What problems, if any, did you encounter?}
   	\\The address being called for our testing environment of the widget could not be used with https:// because the ONID webspace provided is not secure. Just had to change it to http://
   	\item \emph{How long did each task require?}
   	\\1 hour
   	\item \emph{What is the current status (implemented? tested?)}
   	\\Implemented in a test environment but programmed in.
   	\item \emph{What is left to be completed?}
   	\\Making it optimal with the different branches, adding images and making it look good. It also needs to be available as a widget button on the browser for ease of access.
   	\item \emph{For each spike and UML sequence diagram that you developed this week, answer the following}
   	\begin{itemize}
   		\item \emph{Was the spike or diagram useful? Why or why not?}
   		\\No, this was very simplistic in nature and just involved programming skill, research, and team feedback.
   		\item \emph{Were there any diagrams that you wish that you had? Why or why not?}
   		\\Maybe a drawing of what we want this to look like but that can wait when we get to visual design in the final weeks.
   	\end{itemize}
   \end{enumerate}

\subsection{At least a basic user interface}
\begin{enumerate}
	\item \emph{Which pair(s) of teammates worked on that user story's tasks?}
	\\The team was involved in providing input into what exactly we wanted the user interface to be, but Robert and Luke were the people to implement such user story through paired programming.
	\item \emph{What problems, if any, did you encounter?}
	\\This user story was fairly simple to implement and therefore did not result in any significant issues.
	\item \emph{How long did each task require?}
	\\About 1 hour
	\item \emph{What is the current status (implemented? tested?)}
	\\Implemented in a test environment but programmed in.
	\item \emph{What is left to be completed?}
	\\This user story specifies the creation of a simple, graphically and artistically driven UI to reinforce a sense of location to ensure that users who navigate it never get lost and therefore there is nothing left to be complete.
	\item \emph{For each spike and UML sequence diagram that you developed this week, answer the following}
	\begin{itemize}
		\item \emph{Was the spike or diagram useful? Why or why not?}
		\\It was not super useful because the tasks were pretty straight forward in terms of coding and therefore only relied on programming skills.
		\item \emph{Were there any diagrams that you wish that you had? Why or why not?}
		\\No, because we already had paper prototypes of just the basic UI we wanted already available.
	\end{itemize}
\end{enumerate}

\subsection{The navigation page for different categories:}
\begin{enumerate}
	\item \emph{Which pair(s) of teammates worked on that user story's tasks?}
	\\Robert and Luke.
	\item \emph{What problems, if any, did you encounter?}
	\\This user story was also fairly simple to implement and therefore did not result in any significant issues.
	\item \emph{How long did each task require?}
	\\½ hour
	\item \emph{What is the current status (implemented? tested?)}
	\\This has been programmed in a testing environment. 
	\item \emph{What is left to be completed?}
	\\This is pretty much complete, but the team will need to integrate it with the rest of the program for cohesion.
	\item \emph{For each spike and UML sequence diagram that you developed this week, answer the following}
	\begin{itemize}
		\item \emph{Was the spike or diagram useful? Why or why not?}
		\\The spike was not very useful because the tasks for this specific user story were pretty straight forward in terms of coding and therefore only relied on programming skills.
		\item \emph{Were there any diagrams that you wish that you had? Why or why not?}
		\\No, because we already had paper prototypes of just the basic UI we wanted already available.
	\end{itemize}
\end{enumerate}

\subsection{A previous button.:}
\begin{enumerate}
	\item \emph{Which pair(s) of teammates worked on that user story's tasks?}
	\\Robert and Luke.
	\item \emph{What problems, if any, did you encounter?}
	\item \emph{How long did each task require?}
	\\½ hour
	\item \emph{What is the current status (implemented? tested?)}
	\\This in the testing stage as more programming will be needed to create more “pages” in which the previous button would become very useful. At this stage we do not have all of the pages, but are able to program test its use.
	\item \emph{What is left to be completed?}
	\\The integration of implementation of the previous button throughout the program and widget pages.
	\item \emph{For each spike and UML sequence diagram that you developed this week, answer the following}
	\begin{itemize}
		\item \emph{Was the spike or diagram useful? Why or why not?}
		\\The spike proved not to be very useful, because this task was fairly simple in which there we previously decided upon the specific purpose and function of this button, avoiding the need for testing of various ways of implementation.
		\item \emph{Were there any diagrams that you wish that you had? Why or why not?}
		\\This was fairly simple and therefore did not require any diagrams but we did already have a paper prototype of the design of the button already available from a previous assignment and collaborative meeting.
	\end{itemize}
\end{enumerate}
	
	
	
	\section{Design Changes and Rationale}
We encountered a schedule change when we were unable to meet in person as a group at our usual location and time.  This is the first occurence in which we deviated from our decided upon schedule.  But, we were able to navigate around this issue by communicating over the Discord App.  Instead of meeting on Saturday as usual, we decided to reschedule our meeting for collaboration to Monday in which we worked on the user stories decided upon last week at the library in our usual location.  Despite our inability to meet on Saturday, we were able to continue strong communication through the Discord app we have been using throughout this term.  This allowed us to address and discuss the splitting up of tasks for this specific assignment and how we would implement future user stories in a timely matter.  
Furthermore, our project customers were available to answer any questions pertaining to the programming, design, function and expectations of the project.  through our mode of communication, Discord.  We didn’t have many questions of great significance.  But, were able to receive their opinion on certain program elements which we felt the customers should have particular say in.  This included small details such as design and visual elements.  We were able to ask questions about the user interface design such as the topic of colors to be used in correspondence of categories/questions.  They responded very quickly to this specific question as well as others by describing in detail what they wanted or giving visual representations.  In regards to a color scheme, they provided a visual layout of colors they wished to implement and how they would be useful in user navigability.  We decided upon primary colors to be used in varying shades in accordance to categories and questions that would subsequently branch off.

	\section{Tests}
One test that we did was to make sure our formatting was done correctly. In order to do this, we loaded the website through multiple web browsers and devices. The website should work correctly for all supported devices such as Windows, MacOS, iOS, and Android. This was our unit test for a major feature (multi-platform support). We also had to test that our javascript was working correctly. Our solutions come up in the form of a modal in our current prototype. This modal must be able to be hidden and appear correctly to be implemented correctly. If the modal always appeared, the website would appear incorrectly. One issue we ran into is that our modals would not appear on mobile devices. We have decided that for next week, among many other updates, we will either working on fixing this issue, or switch the design away from modals all together. This was our unit test for one not completed feature.
The flex of our website was another issue we worried about. If the size of the web browser were to shrink, how would the icons, panels, and text react? We used a flex framework to help with this. However, we did have to iron out some bugs, such as the fourth icon on the home page jumping to the row below it and making its own row. This issue has been resolved. Overall, we will continue to test our app as we implement new features and new paths
For a specific part of programming this past week, we test our system by clicking “Networking and Wi-Fi” button, then there are four options shown on the screen. These options are either consecutive categories or questions that help the use navigate and discern the specific problem they need help with.  The four options are respectively: “Help with wireless networks”, “Help with wired network”, “Where is the internet down” and “Connect to network drive”. When clicking the first option which is “help with wireless networks”, the website redirects to another step, there are also four options, “My wireless connection is unstable”, “My computer/phone lost connection to the internet”, “I want to setup a wireless network”, “What is VPN? How do I set it up?”. And every option above should return a solution. But now it is now completely implemented, so, the solutions are not shown.
For testing account troubleshooting tree, we click “Account” button on the home page, then there are 5 options, “new students”, “current students”, “former students”, “students employees” and “other employees”.  This specific user story is not completely implemented, so only “”current students” can be clicked and redirect to another page. And then, there are two options shown, “How do I change my ONID password?”, and “How do I set up/change my alternate contact information?”.  When clicking on the first option, “How do I change my ONID password?”, there is a pop up window showing the content that should be solution of the specific issue. But, the contents are not yet implemented for this week.
There are a few cases not completely implemented in the system now. For example, there is no button for returning to homepage or previous steps. The buttons should be at the bottom of the pages, and these may be implemented.



\section{Meeting Report}
We were not able to meet in person due to other obligations this week and sickness but we still maintained contact over Discord. Our team referred to the previous assignment’s user stories for determining what we were going to have accomplished this week in the prototype build of our program. Robert and Luke worked on the programming for the prototype at the library when their schedules aligned, Donghao worked in the tests section, Lauren wrote the User Stories and Development Changes that went into effect, and Evan did the meeting report and LaTex conversion and helped wrote a couple User stories. We all tested out the widget in its current state. From here we can probably get a lot more features implemented next week given we have this solid foundation to work with. 
Next week hopefully we can meet up but this week shows that we can still work well remotely. It also shows that we’ve put a good deal of work into pre production and now have a guideline that we can follow for what remains of the term. This week taught us how to go from planning to actual development with the testing as well as comparing our goals to what we were actually able to get done. Looking at the user stories and how we implemented things are looking good. The development changes will probably happen but this is a simple functional program that we’ll tweak in the next couple of weeks. The test will probably be our main focus as they have the data organized in a correlational way. Things are looking promising and we are on the steps to developing something that could be completely functional at the end of this class. 





\end{document}